% Created 2018-11-20 Tue 17:34
% Intended LaTeX compiler: pdflatex
\documentclass[11pt]{article}
\usepackage[utf8]{inputenc}
\usepackage[T1]{fontenc}
\usepackage{graphicx}
\usepackage{grffile}
\usepackage{longtable}
\usepackage{wrapfig}
\usepackage{rotating}
\usepackage[normalem]{ulem}
\usepackage{amsmath}
\usepackage{amssymb}
\usepackage{textcomp}
\usepackage{amssymb}
\usepackage{capt-of}
\usepackage{hyperref}
\author{dustin}
\date{\today}
\title{}
\hypersetup{
 pdfauthor={dustin},
 pdftitle={},
 pdfkeywords={},
 pdfsubject={},
 pdfcreator={Emacs 25.3.2 (Org mode 9.1.7)}, 
 pdflang={English}}
\begin{document}

\section*{Budgeter}
This program is designed to provide the user with an optimized budget for the
items given. The user should provide a list of items in the form
\begin{center}
\begin{tabular}{ccccc}
Item Name & Importance & Min cost & Max cost & Needed\\
"String" & 1-10 & >0 & > Min cost & True/False\\
\end{tabular}
\end{center}
As well as a the number of year until retirement and the total saved and or
owed.
The algorithm will find the budget allocation that will maximize total value
\begin{align*}
  V_t = \sum_i V_i
\end{align*}
where each value \(V\) is determined by the function.
% \begin{align*}
%   V = -\frac{I}{10} \left( \frac{1}{k(x-m)+0.1} + 10 \right)
% \end{align*}
\begin{align*}
  V &= -I((k(x-max))^2-1) \\
  V &= -I(k(x-max))^2-I \\
  V &= -Ik^2(x-max)^2-I \\
  V &= -Ik^2(x^2-2(max)x+(max)^2)-I \\
  V &= -Ik^2x^2+2Ik^2(max)x-Ik^2(max)^2-I \\
\end{align*}
Where x is the price, I is the importance of the item given by the user,
\(k=\frac{1}{max-min}\), and \(m\) is the maximum price. The derivative in
regard to price is therefore
\begin{align*}
  \frac{dV}{dx} &= -2Ik^2x^2 + 2Ik^2(max)
\end{align*}
The restriction would be
\begin{align*}
  Income &= \sum_i x_i
\end{align*}
The solution of which can be found using the method of langrange multipliers
\begin{align*}
  \mathcal{L} &= f(\vec{x}) - \lambda(g(\vec{x} - b)
\end{align*}
Assuming I only have two costs
\begin{align*}
  Income &= x_1 + x_2 = g(x_1,x_2) = b
\end{align*}
and
\begin{align*}
  V_t &= V_1 + V_2
\end{align*}
At the critical points
\begin{align*}
  \nabla \mathcal{L} &= 0
\end{align*}
Which would lead to a set of linear equations
\begin{align*}
  -2Ik_1^2x_1 + 2Ik_1^2(max_1) &= \lambda \\ 
  -2Ik_2^2x_2 + 2Ik_2^2(min_2) &= \lambda \\ 
  x_1 + x_2 &= X_{total}
\end{align*}
which can be simplified to
\begin{align*}
  -2I_1k_1^2x_1 - \lambda &=  -2I_1k_1^2(max_1)\\ 
  -2I_2k_2^2x_2 - \lambda &=  -2I_2k_2^2(max_2) \\ 
  x_1 + x_2 &= X_{total}
\end{align*}
Which corresponds to the linear equation
\begin{equation}
\begin{bmatrix}
  -2I_1k_1^2 & 0 & -1\\
  0 & -2I_2k_2^2 & -1\\
  1 & 1 & 0\\
\end{bmatrix}
\begin{bmatrix}
  x_1 \\ x_2 \\ \lambda
\end{bmatrix}
=
\begin{bmatrix}
  -2I_1k_1^2(max_1) \\ -2I_2k_2^2(max_2) \\ X_{total}
\end{bmatrix}
\end{equation}
The solution for this equation is easily found and which can easily be
generalized.
\\
Note that an item may be such a bad deal, that the algorithm gives its value a negative
number. Of course you can't have a negative number so the item(s) will be
temporarily removed and the algorithm ran again.  The items will later be added
to final result but with their value over needed set to 0.
\newpage
\subsection*{Test Case Without Savings}
Lets assume we have two items
\begin{itemize}
\item ``efficient'' 10 0 1000 1
\item ``average 5 0 3000 1
\item ``inefficient'' 1 0 5000 1
\end{itemize}
We have a total cost of 4000.
Lets calculate our $k's$
\begin{align*}
  k_1 &= 0.001 \\
  k_2 &= 3.33e-4 \\
  k_3 &= 2e-4
\end{align*}
Our equation should then look like
\begin{equation}
\begin{bmatrix}
  -2(10)(0.001)^2 & 0 & 0 & -1\\
  0 & -2(5)(3.33e-4)^2 & 0 & -1\\
  0 & 0 & -2(10)(2e-4)^2 & -1\\
  1 & 1 & 1 & 0\\
\end{bmatrix}
\begin{bmatrix}
  x_1 \\ x_2 \\ \lambda
\end{bmatrix}
=
\begin{bmatrix}
  -2(10)(0.001)^2(1000) \\ -2(5)(3.33e-4)^2(3000) \\ -2(1)(2e-4)^2(5000) \\4000
\end{bmatrix}
\end{equation}

\begin{equation}
\begin{bmatrix}
  -2e-5 & 0 & 0 & -1\\
  0 & -1.1e-6 & 0 & -1\\
  0 & 0 & -8e-7 & -1\\
  1 & 1 & 1 & 0\\
\end{bmatrix}
\begin{bmatrix}
  x_1 \\ x_2 \\ x_3 \\ \lambda
\end{bmatrix}
=
\begin{bmatrix}
  -0.02 \\ -3.32e-3 \\ -4e^-4 \\ 4000
\end{bmatrix}
\end{equation}
Answer = [989,2804,206], Yay!

\subsection*{Special treatment for needed}
If an object is ``needed'' then the minimum value of the budget item should be
removed for income, and the min and max should be set to
\begin{align*}
  min &= 0 \\
  max &= max-min
\end{align*}
The $min$s should be added the vector outputted by calculatebudget

\newpage
\subsection*{Determining Savings}
The primary goal of this calculation is to assure that at retirement that you
can survive while paying your current costs. This will happen when you are able
to live off the interest of your investment. For this to workout, the future
value of your savings/debt + savings/year need to be equal to the (costs/year)/apr
The future value can calculated for each future year taking the 

\begin{align*}
  V_1 = V_0(1+r) + S
\end{align*}
Where $V_1$ is the current value of savings in year 1, $V_0$ is the value of the savings
year 0 and $S$ is the savings over the year, and $r$ is the annual interest rate.


\subsection*{Choosing importance}
Importance should be scored either in range from 1-10 or from 1-100. A good rule
of thumb is that the score should roughly equal to the percentage of your time
that the effects of the item \textbf{linger} throughout your life. Lets do a few
examples, I will use a scale from 1-10
\subsubsection*{Food}
You may spend anywhere from $30\,min$ to $2\,hr$ eating food throughout the day
which means would mean that you would score it a 1 if I didn't include the word
linger. Though you don't spend a lot of time eating, the nutritional effects of
the food you eat will stay with you the entire day, $100\%$ of your life. While
the ``feel good'' effect of a delicious meal will last a few hours after you
eat. We can divide food into two subsections with their importance values: taste=5, nutrition=10.
Nutrition can be treated as a needed primary price, while taste can be treated
as a premium. A simpler approach would be to average the two numbers to get
food=8, and set a price range appropriate for delicious healthy food.
\subsubsection*{Transportation}
Similar to food, transportation can be separated into a few categories:
Reliability, Convenience, and Status. Reliability affects my entire life and I
would thus score it a 10. Convenience features only affects me while I drive, or
about and hour a day, so I'll score it a 1. Status, for me, will only affect me
in extremely rare occasions, so it'll also give me a 1. Note that I'm married.
For those looking for a partner, the status may possibly get more weight since
the likelihood of the affects could linger increases. Since the importance of
reliability so dramatically surpasses the other considerations, I'll only
consider it during my purchase. Transport:10, min-max based on reliability.
\subsubsection*{Phone}
I'll split the phone into parts: Coverage, Features, and Status.
I give coverage a score of 5, since it's usually not a big deal if I loose
signal while traveling, since I'm usually with others. Features gets a 10, since
I need access to email, text messages, video phone calls, and navigation, for
work. Status receives a 1 for reasons already discussed. Nearly all phones have
good coverage, so my focus will only be on the available features and status.
Status is often cheap with phones on a per/month basis compared to your other costs.
\subsubsection*{Housing}
This has three major parts: Location, size, and status. Location will get a 10,
since it will affect your entire life: education for your kids, commute to work,
neighbors, nearby restaurants etc. The size of the house would get a 5, since I
spend roughly half my time inside my house, while status once again would get a 1.

My input would thus look something like this
\begin{center}
\begin{tabular}{lcccc}
  After-Tax Monthly Income= 1750 &&&&\\
  Saved/Owed= 5000 &&&&\\
  Years till retirement= 25 &&&&\\
  APR = 0.05 &&&&\\
  Item Name & Importance & Min cost & Max cost & Needed\\
  Food(Nutrition) & 10 & 200 & 500 & 1 \\
  Food(Taste) & 5 & 0 & 200 & 0 \\
  Transportation(Reliability) & 10 & 250 & 400 & 1 \\
  Transportation(Status) & 1 & 0 & 500 & 1 \\
  Phone(Features) & 10 & 90 & 120 & 1 \\
  Phone(Status) & 1 & 0 & 30 & 0 \\
  Housing(Location) & 10 & 400 & 1200 & 1 \\
  Housing(Size) & 5 & 0 & 4000 & 0 \\
  Housing(Status) & 1 & 0 & 2000 & 0 \\
\end{tabular}
\end{center}
\end{document}